\subsection{Perceived drawbacks}

We argued the non-goalness of NGO modelling is its strength. Quite the opposite, it can be perceived as vague and risky: what is a model without a hypothesis supposed to give as a result? This appearance seems only augmented by the apparent clarity of explicit hypothesis testing. Nevertheless, we argue that NGO modelling is not more or less risky than any goal-oriented modelling approach, at least within the realm of bottom-up, mechanistic models. The main issue is how to think about model output. Hypothesis-driven research is supposed to clearly state under which analysis result or simulation dynamics the hypothesis is rejected (and under which conditions it is not). In NGO modelling, we have to envision plausible outcome scenarios, why these are likely to arise, and their meaning for our current state of knowledge. Whether such scenarios will indeed occur during the model study---and if we can correctly predict the outcome of the model, maybe a model was not really needed in the first place---is irrelevant, the aim is to imagine the bigger picture in a few qualitatively distinct ways.

For instance, in a fictional model on transcriptional regulation, one may sketch a scenario in which few high affinity binding sites are predicted to be found for housekeeping genes, and many low affinity ones next to environmentally dependent genes. Why? Because low affinity sites are easy to evolve and environmental responses change more often on an evolutionary time scale. Similarly, one can answer what impact such an outcome would have on our understanding of binding site composition and gene function.