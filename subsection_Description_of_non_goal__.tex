\subsection{Description of non-goal oriented modeling}

Non-goal oriented modelling is an approach to modelling -- a modelling philosophy, not a specific formalism. It can be applied regardless of the type of model that one uses, from ordinary differential equations to rule-based frameworks. The core idea is to formulate models which maximize the potential for discovering new insights about, and developing new conceptualization of complex biological systems. The strategy for achieving this is to minimize a priori conceptions on the system of interest by following several guidelines which we discuss in more detail below: (i) No goal: the biological phenomenon of interest is not directly prespecified in the model but self-organizes from interactions between lower level units; (ii) Rich models: the non-goal approach typically necessitates detailed models with (potentially) many degrees of freedom, where lower level units and their interactions are specified at sufficient detail; (iii) Pattern-detection heuristics: the rich dynamics of rich models require methodologies that can  detect interesting emergent features and elucidate their origins.

  