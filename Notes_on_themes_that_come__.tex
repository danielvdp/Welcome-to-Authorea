Notes on themes that come up in the text (hopefully making them explicit will help to clarify things):
\begin{enumerate}
\item Formal models and types of questions: in parts of the text it seems that we are arguing that certain kinds of models are good for certain kinds of questions.
\item Formal models and understanding the world: our underlying theme is that we use models to understand the world around us. This involves answering different kinds of questions and making scientific discoveries. NGO models typically apply to the latter, and are useful for exploring `possible worlds'.
\item Formal models and hypotheses: an NGO approach tends not to have a specific hypothesis (i.e. not hypothesis-driven), but is exploratory and unsupervised. However, often formal models are formulated in order to assess an existing hypothesis (hypothesis-driven). We argue that hypothesis-driven research is biased to preconceived ideas (underlying assumptions). There are two extreme reactions to this: (i) hypothesis-driven research is fine, but we should double-check our assumptions, and (ii) in the face of biological complexity, our preconception are almost certainly flawed and hypothesis-driven research is erroneously biased. In either case, NGO can help us further.
\item Formal models and scientific discovery: formal models can lead to scientific discovery when they lead to new insights about how certain phenomena can arise (e.g. proofs of principle). This is where our understanding of reality is `put on trial', and models help us elucidate counter-intuitive dynamics that cannot be thought-through very easily. NGO models can help us to make such discoveries.
\item Formal models: statistical versus conceptual. We call statistical models those which are formulated in a 1-1 relationship with a particular data set, in order to fit the model quantitatively to the data and either (i) test a particular hypothesis, or (ii) select the best fitting model. The purpose of this exercise is to make an inference about the underlying causation that generates patterns in the data. This statistical process in limited by our ability to `come up with' models or hypotheses (Note that from this statistical perspective 'models' are always a formal instantiation of an 'hypotheses'). We call conceptual models (or phenemonological) those which are formulated in a 1-n relationship (including 1-1) with empirical evidence (including a particular data set), in order to specify a system, of which the dynamics can then be studied. This purpose of this exercise can include multiple aspects including: (i) abstraction from a particular case to demonstrate general principles, (ii) understand the dynamics of (complex) systems, (iii) discover general principles by comparing (complex) systems (i.e. paradigm systems). The dynamics or proofs conceptual models provide us with new insights, search images and predictions which can then be used in statistical modeling. Note that hypotheses can be a starting point, or product, of conceptual models, and hence models can exist as a dynamical system about which hypotheses can be forumulated.
\end{enumerate}
  
  