\subsection{Pattern-detection heuristics}

NGO models and their rich dynamics may be viewed as constituting a little world of their own. It is the responsibility of the modeller to understand this world: how its (predefined) rules lead to different behavioural regimes, and how these relate to our current knowledge. In many ways, the analysis of an model's behaviour mirrors the analysis of experimental data: similar heuristics, analysis methods, and visualization tools may be applied.

First, graphical visualization tools of model dynamics are often indispensable, especially to gain an initial understanding of a model's behaviour. Simple visualization of distributions (histograms), spatial patterns, and time plots help to discover trends along axes of interest. Given the exploratory nature of NGO models, a natural way to analyze their outcomes is by using unsupervised pattern detection methods, like k-means clustering, hierarchical clustering, principal component analysis, and self-organising maps [Kohonen].

Second, the initial exploration of a model's behaviour generates hypotheses regarding the model dynamics and how they originate. Now, tailored-made heuristics can be formulated and investigated. Here the advantage of a model over experiments is that one can re-run models, study the nitty-gritty details, and record practically any observable feature of a model. Thus, once interesting patterns have been discovered, analysis proceeds very much in a hypothesis-driven way. This may seem paradoxical. The important point, however, is that the initial model dynamics have been generated in a hypothesis-free fashion.
