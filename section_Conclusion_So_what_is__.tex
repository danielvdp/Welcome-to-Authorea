\section{Conclusion}

So what is the heuristic value of NGO modelling? Here we can steal some ideas from sys bio approaches. Why do they model the whole system? What does NGO add to the repertoire of a modeller? How does it fit with other modelling approaches? 

Maybe we can try to have a few (three?) core selling points: (1) NGO allows us to determine in an unbiased fashion what are important components+interactions of a system, (2) NGO explicitly fosters concept formation and \emph{new} knowledge generation (unknown unknowns), and (3) NGO complements hypothesis-driven approaches (both analytical and simulation-based) by preceding them and/or as a sanity check (does our current knowledge really support the hypothesis-driven model or did we oversee something?)