\section{Conclusion}

So what is the heuristic value of NGO modelling? In many ways, biology is still a matter of observing and measuring, simply because that is already an extremely challenging task. However, what to measure and where to observe should be done in an informed manner. Also, as biological systems are complex webs of components and their interactions, which often feed back on each other, our brains are just not enough to soundly reason about the consequences of what we observe and measure. We believe the NGO exploratory style of modelling is extremely fruitful for helping to deepen our understanding, and explore the consequences of what we already know (or what we think we know). In other words, NGO modelling explicitly fosters concept formation and \emph{new} knowledge generation. By encouraging to have as few preconceptions as possible on when and where a phenomenon of interest should be found, NGO modelling allows us to determine in an unbiased fashion what are the important components and interactions of a system. In this manner NGO complements hypothesis-driven approaches (both analytical and simulation-based) by preceding them, or as a sanity check (does our current knowledge really support the hypothesis-driven model or did we oversee something?).

With an increasing number of scientists in biology embracing a systems approach, we hope to find that over time they will complement their experiments, data analyses, and hypothesis-driven models with NGO modelling in order to probe their biological system of interest.