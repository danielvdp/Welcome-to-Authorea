\subsection{Perceived drawbacks}

The non-goalness of NGO may be perceived as vague and risky: what is a model without a hypothesis supposed to give as a result? This appearance seems only augmented by the apparent clarity of explicit hypothesis testing. Nevertheless, we argue that NGO modelling is not more or less risky than any goal-oriented modelling approach, at least within the realm of bottom-up, mechanistic models. The main issue is how to think about model output. Hypothesis-driven research is supposed to clearly state under which analysis result or simulation dynamics the hypothesis is rejected (and under which conditions it is not). In NGO modelling, we have to envision plausible outcome scenarios, why these are likely to arise, and their meaning for our current state of knowledge. EXAMPLE NEEDED
