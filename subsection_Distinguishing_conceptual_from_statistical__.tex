\subsection{Distinguishing conceptual from statistical models & introducing hypthesis driven}

With the increased relevance and prevalence of modelling, it is useful to consider under what conditions we can learn something from a model and what it is that we learn. We focus our argument on so-called  conceptual models, which are a class of mathematically and/or rule-based models that can be used to learn something about biology. We exclude statistical modelling, which is used to test existing hypotheses and predictions. Most readers probably agree that we should aim to maximize the opportunities for learning from conceptual models. However there is strong disagreement on how to achieve this. Mainstream theory on modelling mirrors the general idea that science should be strictly hypothesis-driven [Servedio2014]. Thus a model is constructed with a clear goal in mind, namely to describe a certain (biological) phenomenon such that one or more hypotheses can be illustrated and/or later on experimentally tested.
