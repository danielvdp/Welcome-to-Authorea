\subsection{Perceived drawbacks}

The non-goalness of NGO may be perceived as vague and risky: what is a model without a hypothesis supposed to give as a result? This appearance seems only augmented by the apparent clarity of explicit hypothesis testing. Nevertheless, we argue that NGO is not more or less risky than any goal-oriented modelling approach, at least within bottom-up or mechanistic models. 

And instead of formulating a decisive hypothesis that will be tested by the model, in NGO modelling one should envision plausible scenarios of side-effects, or plausible reasons why interesting side-effects are likely to arise. By elaborating on such scenarios and their meaning for our current state of knowledge, one at least covers the known unknowns, which is as much as anyone can do in forecasting science.

WHAT ABOUT THE FACT THAT NGO MODELS ARE RATHER TIME-CONSUMING TO ANALYSE?