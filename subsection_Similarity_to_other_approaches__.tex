\subsection{Similarity to other approaches}

As computers have become more powerful over the last 2-3 decades, their use for both analytical and simulation models has increased. Due to the increase in computational power, there is a trend to create bigger models, more detailed simulations, and WHAT ELSE? 

many of the above mentioned features of NGO models have become popular in a subfield of system biology, where one describes a system and its dynamics at a specific spatial and temporal level by defining it in terms of the components at a lower level [LeNovere]. An important difference between NGO and such system biology models is that often system biology models test specific hypotheses. 

With the increasing computational power, another popular approach is to enumerate all variants of a biological system and explore their behaviour. This exploration of an atlas of a given system is basically NGO exploiting modern day technological possibilities.

