\subsection{Similarity to other approaches}

Modelling without a goal is what sets the NGO approach apart from other modelling techniques. Rich models, however, are found in other research areas of biology, such as ecology, development and systems biology. As computers have become more powerful over the last half century, their use for both analytical and simulation models has increased. This increase in computational power has allowed for bigger models, more detailed simulations, and mechanistic models instead of phenomenological ones. In ecology, this has resulted in initiatives to model ecosystems at planetary scale [REF]. In systems biology not only are models big, there is also the notion that one describes a system and its dynamics at a specific spatial and temporal level by defining it in terms of the components at a lower level [LeNovere]. However, an important difference between NGO and many systems biology models is that in systems biology specific hypotheses are tested. There is an exception, though. Several groups have taken the approach to enumerate all variants of a biological system and explore their behaviour [REFS]. This systematic exploration of an atlas of variants matches NGO in its non-goalness.

WHAT ELSE?