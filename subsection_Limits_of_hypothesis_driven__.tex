\subsection{Limits of hypothesis-driven and knowledge creation via `expectation violation'}

Ironically however, learning from conceptual models often precedes hypothesis testing precisely because the  ideal of hypothesis-driven research fails in practice. Already before any hypothesis is tested, it is often found that models and simulations do not match the modellers’ baseline expectations (or experimental data). It is in those moments that we start learning something truly new. Instead of testing a hypothesis, the models expose gaps in our knowledge and assumptions, enable us to generate new questions, or even enable us to suggest a novel way of conceptualizing  a biological phenomenon. 
  
  