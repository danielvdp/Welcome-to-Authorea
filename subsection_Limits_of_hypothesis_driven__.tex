\subsection{Limits of hypothesis-driven and knowledge creation via `expectation violation'}

Ironically learning from conceptual models often precedes hypothesis testing, precisely because the ideal of hypothesis-driven research fails in practice. Already before any hypothesis is tested, it is often found that models and simulations do not match the modellers’ baseline expectations (or experimental data). It is in those moments that we start learning something truly new. Instead of testing a hypothesis, the models expose faulty assumptions and gaps in our knowledge, enable us to generate new questions, or even suggest a novel way of conceptualizing a biological phenomenon. 

In effect, the original research question has been replaced with a more fundamental one: why is the system not organised as we think it is, but differently? To answer this question, we have to look at differently organised systems, which is not possible if we \emph{a priori} force a model to behave in one particular way. In other words, there is an important class of questions that can only be meaningfully answered by studying and interpreting the overall behaviour of a model, instead of narrowly focussing on a single hypothesis.