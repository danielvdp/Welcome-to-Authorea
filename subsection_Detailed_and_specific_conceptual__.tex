\subsection{Detailed and specific conceptual models}

Highly abstract models and concepts can also be useful for more detailed and specific conceptual modelling. Such models come closer to empirical details, but are still qualitative and abstracted. The point is not to make predictions but to explore the dynamics of complex systems that are representative of specific natural systems. Examples include: how age-structure affects population dynamics (de Roos), how energy-budgets affect evolution (DEB), how spatial-structure and self-structuring affects evolutionary processes (Hogeweg), application of optimility theory to specific kinds of behavior such as anti-predator vigilance, application of game-theory to evolution of conflict (Maynard-Smith) or social learning (Rendell), using spatially extended agent-based models and the concepts self-organization and emergence to study social strcuture (Hogeweg & Hesper) in animal groups or learning and cultural processes (van der Post & Hogeweg), using CA-like models to study the evolution of cooperation or social learning.
Within such conceptual modeling, NGO is a particular subset that aims to maximize the discovery of model dynamics that are counter intuitive. We describe this in more detail below.
  