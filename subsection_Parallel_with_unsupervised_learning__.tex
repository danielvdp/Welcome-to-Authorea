\subsection{Parallel with unsupervised learning methods}

Now that we have introduced the basic ideas behind Non-Goal Oriented modelling, it is illustrative to draw a parallel with the difference between supervised and unsupervised learning algorithms in the field of machine learning and multivariate data analysis [Hogeweg]. Supervised learning can be regarded as goal-oriented. The basic principle of supervised learning is to predefine the categories your data should fall into and subsequently to fit the data into these categories. Thus one imposes \emph{a priori} conceptualizations and/or categorizations on the data. On the other hand, in unsupervised learning one does not assume any categorisation beforehand. Instead, unsupervised learning methods allow us to discover which kinds of categories there are in the data. One ``lets the data speak'' in a rather unbiased fashion. 

Here we argue that hypothesis-driven modelling may be thought of as supervised, while an NGO approach in modelling is alike unsupervised learning: one lets the model speak as unbiasedly as possible. Consequently, the observer has to explore and analyse the model’s behaviour and discover what patterns arise and which model components and interactions underlie that behaviour.