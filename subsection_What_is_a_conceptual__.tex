\subsection{What is a conceptual model? They are abstractions, some analytical, some simulation-based}

Before going into more detail about NGO, we first elaborate on what we mean with conceptual modelling. With conceptual modelling we refer to abstract models that are qualitatively, but not quantitatively, related to empirical findings. These models tend to capture a class of phenomena, instead of a single, specific biological system.

Abstract models come in various flavours. There are population dynamic models, such as the Lotka-Volterra model (REF) that rather straightforwardly capture birth, death, and ecological interactions (predation in the case of Lotka-Volterra). From such models concepts were derived such as carrying capacity (REF), competitive exclusion (REF), oscillations in population sizes, and complex attractors (REF). Another well-known example are population genetic models. These take a bare-bones approach to genetic heredity, focusing on allele frequencies of (independent) genes. They have led to the formulation of the Hardy-Weinberg equilibrium (REF), Hamilton’s rule (REF), frequency-dependent selection (REF), and conditions for sympatric speciation. Other abstract models include the incorporation of game theory in biological thinking, origin of life models such as replicator dynamics and the concept of the information threshold (REF), and optimality theory, the marginal value theorem (REF), and the zero-one diet choice rule in behavioural studies (REF). What these approaches have in common, is that they first and foremost employ the toolkit of mathematical analysis; they are analytical models.

With the advent of powerful and ubiquitous computers, simulation-based approaches have become popular as well. Well-known results are the grammar-based descriptions of plant forms (L-systems), the basic principles of self-organisation, emergent properties, and high-dimensional chaos in cellular automata (REFS) and in Boolean regulatory networks (REFS). Subsequently, cellular automata have inspired grid-based methods for the study of tissue behaviour (cellular potts) and development (REFS). Group-level phenomena such as flocking behaviour have been shown in spatially-extended agent-based models (REF). Such models of cooperating individuals also led to the concept of swarm intelligence, for instance in ant-colony problem solving (REF).

Both analytical and simulation approaches share that concepts resulting from these modelling efforts, were derived from abstract models. These concepts subsequently became useful as a way of phrasing our understanding of experimentally tractable systems: they helped to formulate specific hypotheses and to test these with, for instance, specific statistical models.
  