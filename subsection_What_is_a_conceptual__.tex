\subsection{What is a conceptual model? They are abstractions, some analytical, some simulation-based}

Before going into more detail about NGO, we first elaborate on what we mean with conceptual modelling. With conceptual modelling we refer to abstract models that are qualitatively, but not quantitatively, related to empirical findings. These models tend to capture a class of phenomena, instead of a single, specific biological system.

Abstract models come in various flavours. There are population dynamic models, such as the Lotka-Volterra model (REF) that rather straightforwardly capture birth, death, and ecological interactions (predation). From such models various equilibrium concepts were derived, such as carrying capacity (REF), competitive exclusion (REF), stable equilibrium?, limit cycles, and complex attractors (REF). Another well-known example are population genetic models. These embody a bare-bones approach to genetic heredity, focusing on allele frequencies of (independent) genes. They led to the formulation of the Hardy-Weinberg equilibrium (REF), Hamilton’s rule (REF), frequency-dependent selection (REF), and conditions for sympatric speciation. Other high-level approaches include the incorporation of game theory in biological thinking, origin of life models such as replicator dynamics and the information threshold (REF), and optimality theory and the marginal value theorem in behavioural studies (REF) and zero-one diet choice rule (REF). What these approaches have in common is that they first and foremost employ the toolkit of mathematical analysis; they are analytical models.

In contrast, with the advent of powerful computers, simulation-based approaches have become popular as well. From studying the basic principles of self-organisation, emergent properties and high-dimensional chaos in cellular automata (REFS), to applications of grid-based methods (cellular-potts) and vertex-spring models in the study of developmental processes (REF), to (spatially extended) agent-based modelling and group-level phenomena and swarm intelligence (REF). In the last decade, also networks of various kinds (e.g.\ gene regulory, protein-protein interactions, metabolic) have been explored through simulation studies.

What both approaches share, is that concepts resulting from these modelling efforts, were derived from abstract models. These concepts subsequently became useful as a way of phrasing our understanding of experimentally tractable systems: they helped to formulate specific hypotheses and to test these using, for instance, specific statistical models.
  