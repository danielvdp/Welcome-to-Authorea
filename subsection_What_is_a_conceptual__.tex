\subsection{What is a conceptual model? They are abstractions}

Before going into more detail about NGO, we first elaborate on what we mean with conceptual modelling. With conceptual modelling we refer to abstract models that are generally qualitatively, but not quantitatively, related to empirical findings. Examples include population dynamic models, such as the Lotka-Volterra model (REF), with which various equilibrium concepts were derived such as carrying capacity (REF), competitive exclusion (REF), stable equilibrium, limit cycles and complex attractors (REF). Other examples are population-genetic and the Hardy-Weinberg? equilibrium (REF), adaptive dynamics and sympatric speciation?, game theory and frequency dependent selection (REF) and Hamilton’s rule (REF), replicator dynamics and the information threshold (REF), cellular automata and self-organization of higher-level structures and high-dimensional chaos (REFS), cellular-potts model and developmental processes (REF), (spatially extended) agent-based modelling and group-level phenomena and swarm intelligence (REF), optimality theory and the marginal value theorem (REF) and zero-one diet choice rule (REF), network theory?.

In most cases, the concepts that resulted from these modelling efforts, were derived from abstract models. They subsequently became useful as a way of phrasing our (ongoing) understanding of experimentally tractable systems: they help to formulate specific hypotheses and test these using specific statistical models.
  