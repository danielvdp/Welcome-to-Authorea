\subsection{Harnessing the knowledge creation side of models: NGO}

Such pre-hypothesis `lessons’ from conceptual models are often treated as a side-effect of modelling.  Here we argue it could and should be one of the focal points of modelling. To do so, we advocate an underappreciated approach to modelling that places the above mentioned side-effects at center-stage and hence maximizes the potential for discovery (i.e. learning novelty). In line with previous work, we call the approach “non-goal oriented” (NGO) [Hogeweg]. Since there is no goal, ‘side-effects’ become ‘main effects’.  By NGO we therefore mean: studying the dynamics of models in its most unbiased form possible. In what follows, we address the exploratory, and open-ended nature of NGO modelling, its relevance, and how it complements other modelling philosophies.
