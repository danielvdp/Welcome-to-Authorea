\subsection{What are models for?}

As we stated at the start of this manuscript, models are tools for increasing our understanding of the world around us. Understanding means that we want to be able to realise the causes and effects of the organisational principles that we think are operating in our biological system of interest. Moreover, we can ask why systems are organised as they are and not differently. After all, a lot of biological phenomena could be accomplished in many ways, yet we tend to find a small set of recurring principles at work. We can only look at such different systems if we do not a priori force the model to behave in one particular way. In other words, there is an important class of questions that can only be meaningfully answered by simply studying and interpreting the overal behaviour of a model (instead of narrowly focussing on a single hypothesis).
  