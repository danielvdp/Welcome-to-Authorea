\subsection{What are models for / understanding complexity / non-goal oriented / types of questions / statistical out}

With the increased relevance and prevalence of modelling, it is useful to elaborate on what is meant by \emph{understanding the world around us} and how models help us in this effort. We mean to say that we want to realise what is cause and what is effect in our biological system of interest [REF?]. As a result our knowledge of the system increases: we build up a collection of rules, or organisational principles, that we think are at work. In some sense, our knowledge is a model in our mind that we use to reason about the system [REF?]. However, because biological systems are composed of many units interacting in a myriad ways, simple thought-experiments are difficult and our thinking is easily misguided. Hence, formal, written-out models help us reason about the system in a logically sound way.

In this article, we focus our argument on so-called conceptual models, which are a class of mathematical and/or rule-based models that are explicitly focused on aiding us to discover something new about a biological system (TOO VAGUE?). We exclude statistical models, which are used to test existing hypotheses and predictions. Given a conceptual model, it is useful to consider under what conditions we can learn something novel from it and what it is that we learn. And how would we maximize the opportunities for learning in such models? There is strong disagreement on how to achieve this. Mainstream theory on modelling mirrors the general idea that science should be strictly hypothesis-driven [Servedio2014]. Thus a model is constructed with a clear goal in mind, namely to describe a certain biological phenomenon such that one or more hypotheses can be illustrated and, hopefully, later on experimentally tested. We disagree.
