\subsection{Parallel unsupervised learning methods}

An illustrative parallel can be drawn with the difference between supervised and unsupervised learning algorithms in the field of machine learning and multivariate data analysis [Hogeweg]. Supervised learning can be regarded as goal-oriented. The basic principle of supervised learning is to predefine the categories your data should fall into and subsequently to fit the data into these categories. Thus one imposes a priori conceptualizations and/or categorizations on the data. On the other hand, in unsupervised learning one does not assume any categorisation a priori. Instead, unsupervised learning methods allow us to discover new kinds of categories in the data. One “lets the data speak” in a rather unbiased fashion. In similar fashion, we argue that NGO lets the model speak as unbiased as possible. Consequently, the observer has to explore and analyse the model’s behaviour and discover what patterns arise and which model components and interactions underlie that behaviour.