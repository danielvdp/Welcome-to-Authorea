\section{Evolutionary modelling as a special case}

Models of evolution require a slightly specialized treatment, in the sense that evolution in a model means evolution toward a certain goal. In this case, we believe that the most fruitful approach is to specify as a goal no more than a simple requirement for the actual phenomenon one wants to study. As an example: in case of in silico evolution models, an evolutionary goal needs to be defined. If we take mutational robustness to be the phenomenon of interest, one might think the evolutionary goal should be the maximum level of robustness. However, this only makes sense if one is wondering whether robustness can in fact evolve. In most cases, by directly selecting for robustness, it becomes trivial that it is the outcome of your model simulation. Moreover, you’ll find robustness exactly as you defined it and -- barring any side-effects -- you will not learn anything (“what goes in comes out”). On the other hand, if we define a simple evolutionary goal such as being fit in a constant environment, we may start to see how evolution solves the problem of remaining fit, i.e. of being robust [Nimwegen1999].
  