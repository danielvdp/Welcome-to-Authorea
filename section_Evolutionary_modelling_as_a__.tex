\section{Evolutionary modelling as a special case}

Within the framework of NGO modelling, models of evolution require a slightly specialized treatment. The issue is that \emph{in silico} evolution requires us to predefine a goal that the modelled population evolves toward. At first sight it may seem correct to specify the evolutionary goal in terms of the research question (see also the example below). However, in the majority of cases, the simulated evolutionary process will easily obtain the goal under many circumstances, and one is left to wonder what has been learned. Instead, as advocated by P. Hogeweg [REF], the `correct' and most fruitful approach is to specify an evolutionary goal that is no more than a simple requirement for the actual phenomenon one wants to study. 

As an example, if we take mutational robustness to be the phenomenon of interest, one might think the evolutionary goal should be the maximum level of robustness. However, this only makes sense if one is wondering whether robustness can in fact evolve. In most cases, by directly selecting for robustness, it becomes trivial that it is the outcome of your model simulation. Moreover, you’ll find robustness exactly as you defined it and -- barring any side-effects -- you will not learn anything (``what goes in, comes out''). On the other hand, if we define a simple evolutionary goal such as being fit in a constant environment, we may start to see how evolution solves the problem of remaining fit, i.e. of being robust [Huynen1994, Nimwegen1999].
