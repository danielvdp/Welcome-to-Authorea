\subsection{Rich models}

The model should be formulated such that there are opportunities for a rich dynamics and hence for exploration of those dynamics. This is most likely in a bottom-up, or mechanistic approach [Hogeweg]: one defines the building blocks (f.i. organisms, individual cells, or genes) and how these may interact (e.g. organisms like to be close to each other and adjust their direction of movement to the others). Next, one explores the dynamics and patterns that form as a result of these interacting components. Usually this part of the study is at a more coarse-grained spatiotemporal scale: it is not the individual that we are interested in, it is the collective behaviour of these individuals. Our experience is that often unforeseen consequences arise. 

Of course, as with all modelling approaches also NGO is about simplification and abstraction. As Turing said in the introduction of his famous paper on morphogenesis [Turing1952], a model is a simplification and idealization, and consequently a falsification. However, a good model has interesting features that help us critically evaluate our current state of knowledge.