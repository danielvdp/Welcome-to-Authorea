\subsection{No goal}

In an ideal research setting, the core idea is to formulate a clear research question (as one should always do), yet not to encode the desired answer explicitly in the model. The model should not have an internal “optimum” that corresponds to one’s preconceptions. Instead, one should encode the general system or process within which the phenomenon of interest manifests itself. And the task is to study the model; in particular if, how and when the model displays behaviour similar to the phenomenon that is observed in experiments or in nature. Non-goal oriented modelling advocates the exploration of the possible dynamics of a model and not to dwell on any of the features that follow directly and trivially from the model specifications. In this manner we break free from many of our assumptions about how and when the phenomenon of interest should manifest itself.
  